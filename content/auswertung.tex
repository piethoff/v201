\section{Auswertung}
\label{sec:Auswertung}
\subsection{Bestimmung der Wärmekapazität des Kaloriemeters}
Da die Wärmekapazität des Kaloriemeters für die Bestimmung der Wärmekapazitäten der hier Untersuchten Stoffe benötigt wird, muss diese zuerst bestimmt werden.
Die für diese Messung erhaltenen Werte befinden sich in Tabelle \ref{tab:t1}:
\begin{table}[H]
	\centering
	\caption{Die gemessenen Daten für das Kalorimeter.}
	\label{tab:t1}
	\begin{tabular}{c S[table-format=3.2]}
	\toprule

	$m_{\text{gesamt}}$     & 561.47{\si{\gram}}    \\
	$m_{\text{x}}$  	    & 285.11{\si{\gram}}    \\
	$m_{\text{y}}$  	    & 276.36{\si{\gram}}    \\
	$T_{\text{x}}$  	    & 294.05{\si{\kelvin}}  \\
	$T_{\text{y}}$  	    & 359.85{\si{\kelvin}}  \\
	$T^{\prime}_{\text{m}}$ & 322.35{\si{\kelvin}}  \\
	\bottomrule
	\end{tabular}
\end{table}
Mit der spezifischen Wärmeleitfähigkeit von Wasser \mbox{$c_w = \SI[per-mode=reciprocal]{4,18}{\joule\per\gram\per\kelvin}$ \cite{waermeleit}}
und \mbox{Formel \eqref{eqn:gefaess}} ergibt sich für das Kalorimeter die Wärmekapazität:
\begin{equation*}
c_gm_g = \SI{338.96}{\joule\per\kelvin}
\end{equation*}
\subsection{Bestimmung der Wärmekapazität von Kupfer}
Das Kupferrohr besitzt eine Masse von:
\begin{equation*}
	m_k= \SI{238.31}{\gram}
\end{equation*}
In der drei Messungen für das Kupferrohr werden folgende Werte aufgenommen:
\begin{table}[h]
    \centering
    \caption{.}
    \begin{tabular}{S[table-format=3.2(0)e0] S[table-format=2.1(0)e0] S[table-format=2.1(0)e0] S[table-format=2.1(0)e0] S[table-format=1.3(0)e0] }
        \toprule
        {$m_w/\si{\gram}$} &       {$T_k/\si{\kelvin}$} &       {$T_w/\si{\kelvin}$} &       {$T_m/\si{\kelvin}$} &       {$c_k/\si[per-mode=reciprocal]{\joule\per\gram\per\kelvin}$}\\
        \midrule
        583.59   & 351.45  & 294.05  & 298.15  &  0.897\\
        575.81   & 338.35  & 294.35  & 296.65  &  0.635\\
        591.64   & 335.35  & 294.15  & 296.75  &  0.795\\
        \bottomrule
    \end{tabular}
\end{table}
Somit ergibt sich ein Wert von
\begin{equation*}
	c_k=\SI{0.776\pm 0.0762}{\joule\per\gram\per\kelvin}
\end{equation*}
für die Wämekapazität des Kupferrohrs.
%Der so erhaltene Wert weicht um $\SI{101.56}{\percent}$ von dem Literaturwert $c_k=\SI[per-mode=reciprocal]{0.385}{\joule\per\gram\per\kelvin}$\cite{waermeleit} ab.
\subsection{Bestimmung der Wärmekapazität von Aluminium}
Das Aluminiumrohr besitzt eine Masse von:
\begin{equation*}
	m_k= \SI{114.44}{\gram}
\end{equation*}
In der drei Messungen für das Aluminiumrohr werden folgende Werte aufgenommen:
\begin{table}[H]
    \centering
    \caption{.}
    \begin{tabular}{S[table-format=3.2(0)e0] S[table-format=2.1(0)e0] S[table-format=2.1(0)e0] S[table-format=2.1(0)e0] S[table-format=1.3(0)e0] }
        \toprule
        {$m_w/\si{\gram}$} &       {$T_k/\si{\kelvin}$} &       {$T_w/\si{\kelvin}$} &       {$T_m/\si{\kelvin}$} &       {$c_k/\si[per-mode=reciprocal]{\joule\per\gram\per\kelvin}$}\\
        \midrule
        583.59   & 343.15  & 294.15  & 296.65  & 1.305 \\
        589.14   & 349.35  & 294.85  & 296.85  & 0.933\\
        576.85   & 352.55  & 294.05  & 296.35  & 0.984\\
        \bottomrule
    \end{tabular}
\end{table}
Somit ergibt sich ein Wert von
\begin{equation*}
	c_k=\SI{1.074\pm 0.116}{\joule\per\gram\per\kelvin}
\end{equation*}
für die Wämekapazität des Aluminiumrohrs.
%Der so erhaltene Wert weicht um $20.94\%$ von dem Literaturwert $c_k=\SI[per-mode=reciprocal]{0.888}{\joule\per\gram\per\kelvin}$\cite{waermeleit} ab.
\subsection{Bestimmung der Wärmekapazität von Blei}
Das Bleirohr besitzt eine Masse von:
\begin{equation}
    m_k = \SI{543.34}{\gram}
\end{equation}
In den drei Messungen für das Bleirohr werden folgende Werte aufgenommen:
\begin{table}[H]
    \centering
    \caption{.}
    \begin{tabular}{S[table-format=3.2(0)e0] S[table-format=2.1(0)e0] S[table-format=2.1(0)e0] S[table-format=2.1(0)e0] S}
        \toprule
        {$m_w/\si{\gram}$} &       {$T_k/\si{\kelvin}$} &       {$T_w/\si{\kelvin}$} &       {$T_m/\si{\kelvin}$} & {$c_k/\si[per-mode=reciprocal]{\joule\per\gram\per\kelvin}$}\\
        \midrule
        581.96   & 349.65  & 294.15  & 295.65    & 0.141\\
        586.61  & 352.45  & 294.05  & 295.75    & 0.154\\
        569.12  & 350.55  & 294.25  & 295.85    & 0.146\\
        \bottomrule
    \end{tabular}
\end{table}
Somit ergibt sich ein Wert von:
\begin{equation}
    c_k = \SI{0.147 \pm 0.004}{\joule\per\gram\per\kelvin}
\end{equation}
\subsection{Bestimmung der Atomwärme}
In den Tabellen \ref{tab:at_ku}, \ref{tab:at_al} und \ref{tab:pb} werden die zuvor berechneten Werte in \si{\joule\per\kelvin\per\mol} umgewandelt und mit Gleichung \eqref{eqn:dubbi} die Wärmekapazität bei konstantem Volume berechnet.
Es werden die Konstanten aus Tabelle \ref{tab:const} verwendet.
\begin{table}[H]
	\centering
	\caption{Die benötigten Daten zur Berechnung für die Molwärme von Kupfer.}
	\label{tab:at_ku}
	\begin{tabular}{c c c c c c}
	\toprule
	$ $ & $1.Messung$ & $2. Messung$ & $3. Messung$ \\
	\midrule
	$c_{\text{k}}$ & 0.897 $\frac{\text{J}}{\text{gK}}$ & 0.635 $\frac{\text{J}}{\text{gK}}$ & 0.795 $\frac{\text{J}}{\text{gK}}$ \\
	$c_{\text{p}}$ & 57.00 $\frac{\si{\joule}}{\si{\mol \kelvin}}$ & 40.35 $\frac{\si{\joule}}{\si{\mol \kelvin}}$ & 50.52 $\frac{\si{\joule}}{\si{\mol \kelvin}}$ \\
	$T_{\text{m}}$ & $298,15 \si{\kelvin}$ & $296,65 \si{\kelvin}$ & $296,75 \si{\kelvin}$ \\
	\midrule
	$C_{\text{V}}$ & 56.27 $\frac{\si{\joule}}{\si{\mol \kelvin}}$ &  39.62 $\frac{\si{\joule}}{\si{\mol \kelvin}}$ &  49.52 $\frac{\si{\joule}}{\si{\mol \kelvin}}$ \\
	\bottomrule
	\end{tabular}
\end{table}
Im mittel ergibt sich daraus für Kupfer folgende Wärmekapazität:
\begin{equation*}
	C_V=(48.47 \pm 4.84)\frac{\si{\joule}}{\si{\mol \kelvin}}.
\end{equation*}
Was einer Abweichung von ungefähr $94\%$ zu $3R$ entspricht.
\begin{table}[H]
	\centering
	\caption{Die benötigten Daten zur Berechnung für die Molwärme von Aluminium.}
	\label{tab:at_alu}
	\begin{tabular}{c c c c c c}
	\toprule
	$ $ & $1.Messung$ & $2. Messung$ & $3. Messung$ \\
	\midrule
	$c_{\text{k}}$ & 1.305 $\frac{\text{J}}{\text{gK}}$ & 0.933 $\frac{\text{J}}{\text{gK}}$ & 0.984 $\frac{\text{J}}{\text{gK}}$ \\
	$c_{\text{p}}$ & 35.235 $\frac{\si{\joule}}{\si{\mol \kelvin}}$ & 25.191 $\frac{\si{\joule}}{\si{\mol \kelvin}}$ & 26.568 $\frac{\si{\joule}}{\si{\mol \kelvin}}$ \\
	$T_{\text{m}}$ & $296,65 \si{\kelvin}$ & $296,85 \si{\kelvin}$ & $296,35 \si{\kelvin}$ \\
	\midrule
	$C_{\text{V}}$ & 34.130 $\frac{\si{\joule}}{\si{\mol \kelvin}}$ &  24.084 $\frac{\si{\joule}}{\si{\mol \kelvin}}$ &  25.463 $\frac{\si{\joule}}{\si{\mol \kelvin}}$ \\
	\bottomrule
	\end{tabular}
\end{table}
Im mittel ergibt sich daraus für Aluminium folgende Wärmekapazität:
\begin{equation*}
	C_V=(27.89 \pm 3.14)\frac{\si{\joule}}{\si{\mol \kelvin}}.
\end{equation*}
Was einer Abweichung von ungefähr $12\%$ zu $3R$ entspricht.
\begin{table}[H]
	\centering
	\caption{Die benötigten Daten zur Berechnung für die Molwärme von Blei.}
	\label{tab:at_pb}
	\begin{tabular}{c c c c c c}
	\toprule
	$ $ & $1.Messung$ & $2. Messung$ & $3. Messung$ \\
	\midrule
	$c_{\text{k}}$ & 0.141 $\frac{\text{J}}{\text{gK}}$ & 0.154 $\frac{\text{J}}{\text{gK}}$ & 0.146 $\frac{\text{J}}{\text{gK}}$ \\
	$c_{\text{p}}$ & 29.215 $\frac{\si{\joule}}{\si{\mol \kelvin}}$ & 31.909 $\frac{\si{\joule}}{\si{\mol \kelvin}}$ & 30.251 $\frac{\si{\joule}}{\si{\mol \kelvin}}$ \\
	$T_{\text{m}}$ & $295,65 \si{\kelvin}$ & $295,75 \si{\kelvin}$ & $295,85 \si{\kelvin}$ \\
	\midrule
	$C_{\text{V}}$ & 27.499 $\frac{\si{\joule}}{\si{\mol \kelvin}}$ &  30.192 $\frac{\si{\joule}}{\si{\mol \kelvin}}$ &  28.534 $\frac{\si{\joule}}{\si{\mol \kelvin}}$ \\
	\bottomrule
	\end{tabular}
\end{table}
Im mittel ergibt sich daraus für Aluminium folgende Wärmekapazität:
\begin{equation*}
	C_V=(28.74 \pm 0.78)\frac{\si{\joule}}{\si{\mol \kelvin}}.
\end{equation*}
Was einer Abweichung von ungefähr $15\%$ zu $3R$ entspricht.
