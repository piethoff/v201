\section{Auswertung}
\label{sec:Auswertung}
\subsection{Bestimmung der Wärmekapazität des Kaloriemeters}
Da die Wärmekapazität des Kaloriemeters für die Bestimmung der Wärmekapazitäten der hier Untersuchten Stoffe benötigt wird, muss diese zuerst bestimmt werden.
Die für diese Messung erhaltenen Werte befinden sich in Tabelle \ref{tab:t1}:
\begin{table}
	\centering
	\caption{Die gemessenen Daten für das Kalorimeter.}
	\label{tab:t1}
	\begin{tabular}{c s[table-format=3.2]}
	\toprule

	$m_{\text{gesamt}}$     & \SI{561.47}{\gram}    \\
	$m_{\text{x}}$  	    & \SI{285.11}{\gram}    \\
	$m_{\text{y}}$  	    & \SI{276.36}{\gram}    \\
	$T_{\text{x}}$  	    & \SI{294.05}{\kelvin}  \\
	$T_{\text{y}}$  	    & \SI{359.85}{\kelvin}  \\
	$T^{\prime}_{\text{m}}$ & \SI{322.35}{\kelvin}  \\
	\bottomrule
	\end{tabular}
\end{table}
Mit der spezifischen Wärmeleitfähigkeit von Wasser $c_w = \SI[per-mode=reciprocal]{4,18}{\joule\per\gram\per\kelvin}$ \cite{waermeleit} 
und Formel \eqref{eqn:gefaess} ergibt sich für das Kalorimeter die Wärmekapazität:
\begin{equation*}
c_gm_g = \SI{338.96}{\joule\per\kelvin}
\end{equation*}
\subsection{Bestimmung der Wärmekapazität von Kupfer}
%
\subsection{Bestimmung der Wärmekapazität von Aluminium}
%
\subsection{Bestimmung der Wärmekapazität von Blei}
In den drei Messungen für das Bleirohr werden folgende Werte aufgenommen:
\begin{table}
    \centering
    \caption{.}
    \begin{tabular}{S[table-format=3.2(0)e0] S[table-format=2.1(0)e0] S[table-format=2.1(0)e0] S[table-format=2.1(0)e0] }
        \toprule
        {$m_w$} &       {$T_k$} &       {$T_w$} &       {$T_m$} \\
        \midrule
        788.6   & 76.5  & 21.0  & 22.5  \\
        793.25  & 79.3  & 20.9  & 22.6  \\
        775.76  & 77.4  & 21.1  & 22.7  \\
        \bottomrule
    \end{tabular}
\end{table}
