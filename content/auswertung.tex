\section{Auswertung}
\label{sec:Auswertung}
\subsection{Bestimmung der Wärmekapazität des Kaloriemeters}
Da die Wärmekapazität des Kaloriemeters für die Bestimmung der Wärmekapazitäten
der hier Untersuchten Stoffe benötigt wird, wird diese zuerst bestimmt.
Die für diese Messung erhaltenen Werte befinden sich in Tabelle \ref{tab:t1}:
\begin{table}[H]
	\centering
	\caption{Die gemessenen Daten für das Kalorimeter.}
	\label{tab:t1}
	\begin{tabular}{SSSSSS}
	\toprule
	{$m_{\text{gesamt}}$/\si{\gram}} & {$m_{\text{x}}$/\si{\gram}} & {$m_{\text{y}}$/\si{\gram}} &
{$T_{\text{x}}$/\si{\kelvin}} &
{$T_{\text{y}}$/\si{\gram}} &
        {$T^{\prime}_{\text{m}}$/\si{\gram}}\\
    \midrule
    561.47 & 285.11 & 276.36 & 294.05 & 359.85 & 322.35 \\
    \bottomrule
	\end{tabular}
\end{table}
\noindent Mit der spezifischen Wärmeleitfähigkeit von Wasser \mbox{$c_w =
\SI[per-mode=reciprocal]{4,18}{\joule\per\gram\per\kelvin}$\cite{waermeleit}}
und \mbox{Formel \eqref{eqn:gefaess}} ergibt sich für das Kalorimeter die Wärmekapazität:
\begin{equation*}
c_gm_g = \SI{338.96}{\joule\per\kelvin}
\end{equation*}
\subsection{Bestimmung der Wärmekapazität von Kupfer}
Das Kupferrohr besitzt eine Masse von:
\begin{equation*}
	m_k= \SI{238.31}{\gram}
\end{equation*}
In den drei Messungen für das Kupferrohr werden folgende Werte aufgenommen 
und mit den zugehörigen Wärmekapazitäten nach Gleichung \eqref{eq:wkap} aufgetragen:
\begin{table}[H]
    \centering
    \caption{Messwerte und Wärmekapazitäten für Kupfer.}
    \label{tab:at_cu}
    \begin{tabular}{S[table-format=3.2(0)e0] S[table-format=3.2(0)e0] S[table-format=3.2(0)e0] S[table-format=3.2(0)e0] S[table-format=1.3(0)e0] }
        \toprule
        {$m_w/\si{\gram}$} &       {$T_k/\si{\kelvin}$} &       {$T_w/\si{\kelvin}$} &       {$T_m/\si{\kelvin}$} &       {$c_k/\si[per-mode=reciprocal]{\joule\per\gram\per\kelvin}$}\\
        \midrule
        583.59   & 351.45  & 294.05  & 298.15  &  0.897\\
        575.81   & 338.35  & 294.35  & 296.65  &  0.635\\
        591.64   & 335.35  & 294.15  & 296.75  &  0.795\\
        \bottomrule
    \end{tabular}
\end{table}
\noindent Somit ergibt sich ein Wert von
\begin{equation*}
	c_k=\SI{0.78\pm 0.08}{\joule\per\gram\per\kelvin}
\end{equation*}
für die Wämekapazität des Kupfer.
Der Fehler berechnet sich mit:
\begin{equation*}
	\symup{\Delta}c_k = \sqrt{\frac{1}{N(N-1)}\sum_{j=1}^N (c_{kj} - \bar{c}_k)^2}
\end{equation*}
%Der so erhaltene Wert weicht um $\SI{101.56}{\percent}$ von dem Literaturwert $c_k=\SI[per-mode=reciprocal]{0.385}{\joule\per\gram\per\kelvin}$\cite{waermeleit} ab.
\subsection{Bestimmung der Wärmekapazität von Aluminium}
Das Aluminiumrohr besitzt eine Masse von:
\begin{equation*}
	m_k= \SI{114.44}{\gram}
\end{equation*}
In den drei Messungen für das Aluminiumrohr werden folgende Werte aufgenommen
und mit den zugehörigen Wärmekapazitäten nach Gleichung \eqref{eq:wkap} aufgetragen:
\begin{table}[H]
    \centering
    \caption{Messwerte und Wärmekapazitäten für Aluminium.}
    \label{tab:at_al}
    \begin{tabular}{S[table-format=3.2(0)e0] S[table-format=3.2(0)e0] S[table-format=3.2(0)e0] S[table-format=3.2(0)e0] S[table-format=1.3(0)e0] }
        \toprule
        {$m_w/\si{\gram}$} &       {$T_k/\si{\kelvin}$} &       {$T_w/\si{\kelvin}$} &       {$T_m/\si{\kelvin}$} &       {$c_k/\si[per-mode=reciprocal]{\joule\per\gram\per\kelvin}$}\\
        \midrule
        583.59   & 343.15  & 294.15  & 296.65  & 1.305 \\
        589.14   & 349.35  & 294.85  & 296.85  & 0.933\\
        576.85   & 352.55  & 294.05  & 296.35  & 0.984\\
        \bottomrule
    \end{tabular}
\end{table}
\noindent Somit ergibt sich ein Wert von
\begin{equation*}
	c_k=\SI{1.07\pm 0.12}{\joule\per\gram\per\kelvin}
\end{equation*}
für die Wämekapazität des Aluminium.
Der Fehler berechnet sich mit:
\begin{equation*}
	\symup{\Delta} c_k = \sqrt{\frac{1}{N(N-1)}\sum_{j=1}^N (c_{kj} - \bar{c}_k)^2}
\end{equation*}
%Der so erhaltene Wert weicht um $20.94\%$ von dem Literaturwert $c_k=\SI[per-mode=reciprocal]{0.888}{\joule\per\gram\per\kelvin}$\cite{waermeleit} ab.
\subsection{Bestimmung der Wärmekapazität von Blei}
Das Bleirohr besitzt eine Masse von:
\begin{equation}
    m_k = \SI{543.34}{\gram}
\end{equation}
In den drei Messungen für das Bleirohr werden folgende Werte aufgenommen
und mit den zugehörigen Wärmekapazitäten nach Gleichung \eqref{eq:wkap} aufgetragen:
\begin{table}[H]
    \centering
    \caption{Messwerte und Wärmekapazitäten für Blei.}
    \label{tab:at_pb}
    \begin{tabular}{S[table-format=3.2(0)e0] S[table-format=3.2(0)e0] S[table-format=3.2(0)e0] S[table-format=3.2(0)e0] S[table-format=1.3(0)e0]}
        \toprule
        {$m_w/\si{\gram}$} &       {$T_k/\si{\kelvin}$} &
        {$T_w/\si{\kelvin}$} &       {$T_m/\si{\kelvin}$} &
        {$c_k/\si[per-mode=reciprocal]{\joule\per\kg\per\kelvin}$}\\
        \midrule
        581.96  & 349.65  & 294.15  & 295.65    & 0.142  \\
        586.61  & 352.45  & 294.05  & 295.75    & 0.154  \\
        569.12  & 350.55  & 294.25  & 295.85    & 0.146  \\
        \bottomrule
    \end{tabular}
\end{table}
\noindent Somit ergibt sich ein Wert von:
\begin{equation*}
    c_k = \SI{0.15\pm0.01}{\joule\per\gram\per\kelvin}
\end{equation*}
Der Fehler berechnet sich mit:
\begin{equation*}
	\symup{\Delta} c_k = \sqrt{\frac{1}{N(N-1)}\sum_{j=1}^N (c_{kj} - \bar{c}_k)^2}
\end{equation*}
\subsection{Bestimmung der molaren Wärmekapazität}
In den Tabellen \ref{tab:at_cu}, \ref{tab:at_al} und \ref{tab:at_pb} werden die zuvor berechneten spezifischen Wärmekapazitäten in molare Wärmekakazitäten umgerechnet und mit Gleichung \eqref{eqn:konstv} die Wärmekapazität bei konstantem Volumen berechnet.
Es werden die Konstanten aus Tabelle \ref{tab:const} und \ref{tab:kapwerte} verwendet.
Die molare Wärmekapazität und ihre Unsicherheit berechnen sich wie folgt:
\begin{gather}
    C_{m,V}= c_kM - (3\alpha)^2\kappa \frac{M}{\rho} T_k\\
    \symup{\Delta}C_{m,V}=\sqrt{\left(M\symup{\Delta}c_k\right)^2+\left((3\alpha)^2\kappa \frac{M}{\rho} \symup{\Delta}T_k\right)^2}
\end{gather}
%
Die Unsicherheit der mittleren Temperaturen $T_k$ berechnet sich wie folgt:
\begin{equation*}
	\symup{\Delta} T_k = \sqrt{\frac{1}{N(N-1)}\sum_{j=1}^N (T_{kj} - \bar{T}_k)^2}
\end{equation*}
\begin{table}[H]
    \centering
    \caption{Werte zur Berechung der molaren Wärmekapazität.}
    \label{tab:kapwerte}
    \begin{tabular}{c S[table-format=1.2(3)] S[table-format=3.2(3)]}
        \toprule
        {Material} & {$c_k/\si[per-mode=reciprocal]{\joule\per\gram\per\kelvin}$} & {$T_k/\si{\kelvin}$}\\
        \midrule
        {Kupfer}    &   0.78\pm0.08 &   341.72\pm4.94 \\
        {Aluminium} &   1.07\pm0.12 &   348.35\pm2.76 \\
        {Blei}      &   0.15\pm0.01 &   350.88\pm0.83 \\
        \bottomrule
    \end{tabular}
\end{table}
\noindent Für Kupfer ergibt sich folgende Wärmekapazität:
\begin{equation*}
	C_{m,V}=\SI{48.69 \pm 5.08}{\joule\per\mol\per\kelvin}
\end{equation*}
Was einer Abweichung von ungefähr $\SI{95.20\pm20.37}{\percent}$ zu $3R$ entspricht.
\\
\noindent Für Aluminium ergibt sich folgende Wärmekapazität:
\begin{equation*}
	C_{m,V}=\SI{27.59 \pm 3.24}{\joule\per\mol\per\kelvin}
\end{equation*}
Was einer Abweichung von ungefähr $\SI{10.61\pm12.99}{\percent}$ zu $3R$ entspricht.
\\
\noindent Für Blei ergibt sich folgende Wärmekapazität:
\begin{equation*}
	C_{m,V}=\SI{29.04 \pm 2.07}{\joule\per\mol\per\kelvin}
\end{equation*}
Was einer Abweichung von ungefähr $\SI{16.42\pm8.30}{\percent}$ zu $3R$ entspricht.
