\section{Auswertung}
\label{sec:Auswertung}
\subsection{Bestimmung der Wärmekapazität des Kaloriemeters}
Da die Wärmekapazität des Kaloriemeters für die Bestimmung der Wärmekapazitäten
der hier Untersuchten Stoffe benötigt wird, wird diese zuerst bestimmt.
Die für diese Messung erhaltenen Werte befinden sich in Tabelle \ref{tab:t1}:
\begin{table}[H]
	\centering
	\caption{Die gemessenen Daten für das Kalorimeter.}
	\label{tab:t1}
	\begin{tabular}{SSSSSS}
	\toprule
	{$m_{\text{gesamt}}$/\si{\gram}} & {$m_{\text{x}}$/\si{\gram}} & {$m_{\text{y}}$/\si{\gram}} &
{$T_{\text{x}}$/\si{\kelvin}} &
{$T_{\text{y}}$/\si{\gram}} &
        {$T^{\prime}_{\text{m}}$/\si{\gram}}\\
    \midrule
    561.47 & 285.11 & 276.36 & 294.05 & 359.85 & 322.35 \\
    \bottomrule
	\end{tabular}
\end{table}
\noindent Mit der spezifischen Wärmeleitfähigkeit von Wasser \mbox{$c_w =
\SI[per-mode=reciprocal]{4,18}{\joule\per\gram\per\kelvin}$ \cite{waermeleit}}
und \mbox{Formel \eqref{eqn:gefaess}} ergibt sich für das Kalorimeter die Wärmekapazität:
\begin{equation*}
c_gm_g = \SI{338.96}{\joule\per\kelvin}
\end{equation*}
\subsection{Bestimmung der Wärmekapazität von Kupfer}
Das Kupferrohr besitzt eine Masse von:
\begin{equation*}
	m_k= \SI{238.31}{\gram}
\end{equation*}
In den drei Messungen für das Kupferrohr werden folgende Werte aufgenommen:
\begin{table}[H]
    \centering
    \caption{Messwerte und Wärmekapazitäten für Kupfer.}
    \begin{tabular}{S[table-format=3.2(0)e0] S[table-format=3.2(0)e0] S[table-format=3.2(0)e0] S[table-format=3.2(0)e0] S[table-format=1.3(0)e0] }
        \toprule
        {$m_w/\si{\gram}$} &       {$T_k/\si{\kelvin}$} &       {$T_w/\si{\kelvin}$} &       {$T_m/\si{\kelvin}$} &       {$c_k/\si[per-mode=reciprocal]{\joule\per\gram\per\kelvin}$}\\
        \midrule
        583.59   & 351.45  & 294.05  & 298.15  &  0.897\\
        575.81   & 338.35  & 294.35  & 296.65  &  0.635\\
        591.64   & 335.35  & 294.15  & 296.75  &  0.795\\
        \bottomrule
    \end{tabular}
\end{table}
\noindent Somit ergibt sich ein Wert von
\begin{equation*}
	c_k=\SI{0.78\pm 0.08}{\joule\per\gram\per\kelvin}
\end{equation*}
für die Wämekapazität des Kupfer.
Der Fehler berechnet sich mit:
\begin{equation*}
	\symup{\Delta c_k} = \sqrt{\frac{1}{N(N-1)}\sum_{j=1}^N (c_{kj} - \bar{c_k})^2}
\end{equation*}
%Der so erhaltene Wert weicht um $\SI{101.56}{\percent}$ von dem Literaturwert $c_k=\SI[per-mode=reciprocal]{0.385}{\joule\per\gram\per\kelvin}$\cite{waermeleit} ab.
\subsection{Bestimmung der Wärmekapazität von Aluminium}
Das Aluminiumrohr besitzt eine Masse von:
\begin{equation*}
	m_k= \SI{114.44}{\gram}
\end{equation*}
In den drei Messungen für das Aluminiumrohr werden folgende Werte aufgenommen:
\begin{table}[H]
    \centering
    \caption{Messwerte und Wärmekapazitäten für Aluminium.}
    \begin{tabular}{S[table-format=3.2(0)e0] S[table-format=3.2(0)e0] S[table-format=3.2(0)e0] S[table-format=3.2(0)e0] S[table-format=1.3(0)e0] }
        \toprule
        {$m_w/\si{\gram}$} &       {$T_k/\si{\kelvin}$} &       {$T_w/\si{\kelvin}$} &       {$T_m/\si{\kelvin}$} &       {$c_k/\si[per-mode=reciprocal]{\joule\per\gram\per\kelvin}$}\\
        \midrule
        583.59   & 343.15  & 294.15  & 296.65  & 1.305 \\
        589.14   & 349.35  & 294.85  & 296.85  & 0.933\\
        576.85   & 352.55  & 294.05  & 296.35  & 0.984\\
        \bottomrule
    \end{tabular}
\end{table}
\noindent Somit ergibt sich ein Wert von
\begin{equation*}
	c_k=\SI{1.07\pm 0.12}{\joule\per\gram\per\kelvin}
\end{equation*}
für die Wämekapazität des Aluminium.
Der Fehler berechnet sich mit:
\begin{equation*}
	\Delta c_k = \sqrt{\frac{1}{N(N-1)}\sum_{j=1}^N (c_{kj} - \bar{c_k})^2}
\end{equation*}
%Der so erhaltene Wert weicht um $20.94\%$ von dem Literaturwert $c_k=\SI[per-mode=reciprocal]{0.888}{\joule\per\gram\per\kelvin}$\cite{waermeleit} ab.
\subsection{Bestimmung der Wärmekapazität von Blei}
Das Bleirohr besitzt eine Masse von:
\begin{equation}
    m_k = \SI{543.34}{\gram}
\end{equation}
In den drei Messungen für das Bleirohr werden folgende Werte aufgenommen:
\begin{table}[H]
    \centering
    \caption{Messwerte und Wärmekapazitäten für Blei.}
    \begin{tabular}{S[table-format=3.2(0)e0] S[table-format=3.2(0)e0] S[table-format=3.2(0)e0] S[table-format=3.2(0)e0] S[table-format=1.3(0)e0]}
        \toprule
        {$m_w/\si{\gram}$} &       {$T_k/\si{\kelvin}$} &       {$T_w/\si{\kelvin}$} &       {$T_m/\si{\kelvin}$} & {$c_k/\si[per-mode=reciprocal]{\joule\per\gram\per\kelvin}$}\\
        \midrule
        581.96   & 349.65  & 294.15  & 295.65    & 0.141\\
        586.61  & 352.45  & 294.05  & 295.75    & 0.154\\
        569.12  & 350.55  & 294.25  & 295.85    & 0.146\\
        \bottomrule
    \end{tabular}
\end{table}
\noindent Somit ergibt sich ein Wert von:
\begin{equation}
    c_k = \SI{0.15 \pm 0.01}{\joule\per\gram\per\kelvin}
\end{equation}
Der Fehler berechnet sich mit:
\begin{equation*}
	\Delta c_k = \sqrt{\frac{1}{N(N-1)}\sum_{j=1}^N (c_{kj} - \bar{c_k})^2}
\end{equation*}
\subsection{Bestimmung der Atomwärme}
In den Tabellen \ref{tab:at_cu}, \ref{tab:at_al} und \ref{tab:at_pb} werden die zuvor berechneten Werte in \si{\joule\per\kelvin\per\mol} umgewandelt und mit Gleichung \eqref{eqn:dubbi} die Wärmekapazität bei konstantem Volumen berechnet.
Es werden die Konstanten aus Tabelle \ref{tab:const} verwendet.
Die Atomwärme und ihre Unsicherheit berechnen sich wie folgt:
\begin{gather}
    C_V=    \\
    \symup{\Delta}C_V=
%
\end{gather}
\noindent Im Mittel ergibt sich daraus für Kupfer folgende Wärmekapazität:
\begin{equation*}
	C_V=\SI{48.47 \pm 4.84}{\joule\per\mol\per\kelvin}.
\end{equation*}
Was einer Abweichung von ungefähr $94\%$ zu $3R$ entspricht.
\noindent Im Mittel ergibt sich daraus für Aluminium folgende Wärmekapazität:
\begin{equation*}
	C_V=(27.89 \pm 3.14)\frac{\si{\joule}}{\si{\mol \kelvin}}.
\end{equation*}
Was einer Abweichung von ungefähr $12\%$ zu $3R$ entspricht.
\noindent Im Mittel ergibt sich daraus für Blei folgende Wärmekapazität:
\begin{equation*}
	C_V=(28.74 \pm 0.78)\frac{\si{\joule}}{\si{\mol \kelvin}}.
\end{equation*}
Was einer Abweichung von ungefähr $15\%$ zu $3R$ entspricht.
