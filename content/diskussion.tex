\section{Diskussion}
\label{sec:Diskussion}
Die Abweichungen vom Dulong-Petitschen Gesetz sind relativ gering für Aluminium,$12\%$, und Blei,$15\%$.
Für Kupfer sind größere Abweichungen, ungefähr $94\%$, zu beobachten.
Dies kann einerseits an einer Fehlerbehafteten Messung liegen andererseits auch an der limitierten Gültigkeit des Dulong-Petitschen Gesetzes.
Bei der durchfüehrung des Versuches ist aufgefallen, dass die Messwerte des Thermometers stark schwanken und zum Teil auch weit entfernt von der Erwartung liegt.
Zusätzlich muss auch in betracht gezogen werden, dass die Aufhängung der Probekörper sich nur geringfügig Abkühlen und somit die Messung der Mischtemperatur beeinnflussen.
Es lässt sich also sagen, dass der Aufbau systematische Fehlerquellen enthält.
Mit den Hier gemessenen Werten kann gefolgert werden, dass das Dulong-Petitsche Gesetz nur bedingt gilt.
Es ist jedoch fragwürdig wie Aussagekräftig die hier gemessenen Werte sind.
