\section{Zielsetzung}
Zielsetzung dieses Versuchs ist es die Gültigkeit des Dulong-Petitsche-Gesetzes zu üperprüfen.
\section{Theorie}
\label{sec:Theorie}
Wenn ein Körper sich um die Temperatur $\Delta T$ erwärmt, nimmt er die Wärmemenge
\begin{equation}
\Delta Q = m c \Delta T
\end{equation}
auf.
Die spezifiche Wärmekapazität wird durch
\begin{equation}
c_k = \frac{\Delta Q}{m \cdot T}
\end{equation}
errechnet.
Es wird unterschieden zwischen der WäWärmekapazität $c_p$ bei kontanten Druck und $c_v$ bei konstanten Volumen.
\subsection{Das Dulong-Petitsche Gesetz}
Das Dulong-Petitsche Gesetz besagt, dass die Atomwärme $C-V$ im festen Aggregatzustand unabhängig vom chemischen Charakter des Elements konstant $3R$ beträgt.
$R$ ist hier die allgemeine Gaskonstante.
 In der klassischen Physik lässt sich diese Annahme über die innere Energie
 \begin{equation}
 \langle u \rangle = \frac{1}{\tau} \int_0 ^\tau u(t)dt = \langle E_{kin} \rangle + \langle E_{pot} \rangle
 \end{equation}
hergeleitet.
Die kinetische und potentielle Energien lassen sich über die Beobachtung herleiten, dass Atome wie harmonische Oszillatoren schwingen.
Aus den so aufgestellten Gleichungen für die Energien folgt der Ausdruck:
\begin{equation}
C_V = 3R .
\end{equation}
\subsection{Abweichung vom Dulong-Petitsche Gesetz}
Der Wert $3R$ wird bei hohen Temperaturen tatsächlich gemessen.
Die meisten Elemente erreichen diesen schon bei Raumtemperatur.
Substanzen mit geringen Atomgewicht erreichen diesen erst bei $1000\si{\celsius}$.
Dies lässt sich nicht mit der klassischen Physik erklären, daher muss das Problem quantenmechanisch beobachtet werden.
Die Annahme des Dulong-Petitsche Gesetzes, dass atomre Oszillatoren in Festkörpern beliebig kleine Energien aufnehmen und abgeben können wiederspricht der Quantenmechanik.
Diese besagt, dass Oszillatoren ihre Energie nur in Beträgen ändern können:
\begin{equation}
\Delta u = n \hbar \omega .
\end{equation}
Somit lässt sich über die Energie folgende Aussage treffen:
\begin{equation}
\langle U_{qu} \rangle = 3 N_L \langle u_{qu} \rangle = \frac{3 N_L h \omega}{e^{\frac{h\omega}{kT}}-1} .
\end{equation}
Aus der Taylorentwicklung der Exponentialfunktion kann entnommen werden, dass $\langle U_{qu} \rangle$ für hohe Werte von $T$ gegen $3RT$ strebt.
