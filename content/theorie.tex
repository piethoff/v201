\section{Zielsetzung}
Zielsetzung dieses Versuchs ist es die Gültigkeit des Dulong-Petitschen-Gesetzes zu überprüfen.
\section{Theorie}
\label{sec:Theorie}
Wenn ein Körper sich um die Temperatur $\symup{\Delta} T$ erwärmt, nimmt er die Wärmemenge
\begin{equation}
\symup{\Delta} Q = cm \symup{\Delta} T
\end{equation}
auf.
Die spezifische Wärmekapazität wird durch
\begin{equation}
c = \frac{\symup{\Delta} Q}{m\symup{\Delta}T}
\end{equation}
berechnet.
Mithilfe des Ersten Hauptsatzes der Thermodynamik
\begin{equation}
    \symup{d}U = \symup{d}Q + \symup{d}A
\end{equation}
und
\begin{equation}
    \symup{d}A = -p\symup{d}V \stackrel{\text{V=const.}}{=} 0
\end{equation}
kann die Wärmekakazität, bei konstantem Volumen, mit
\begin{equation}
    C_V = \left(\frac{\symup{\Delta}U}{\symup{\Delta}T}\right)_\text{V=const.}
\end{equation}
ausgedrückt werden und somit folgt für die molare Wärmekapazität, auch Atomwärme,
\begin{equation}
    \label{eq:molkap}
    C_{m,V} = \frac{M}{m}\left(\frac{\symup{\Delta}U}{\symup{\Delta}T}\right)_\text{V=const.},
\end{equation}
mit $m$ als Masse des Körpers und $M$ als molare Masse.
\subsection{Das Dulong-Petitsche Gesetz}
Das Dulong-Petitsche Gesetz besagt, dass die molare Wärmekapazität $C_{m,V}$ im festen Aggregatzustand
unabhängig vom chemischen Charakter des Stoffs konstant $3R$ beträgt.
$R$ ist hier die allgemeine Gaskonstante.
In der klassischen Physik leitet sich diese Annahme über die mittlere innere Energie
 \begin{equation}
 \langle u \rangle = \frac{1}{\tau} \int_0^\tau u(t)\symup{dt} = \langle E_\text{kin} \rangle + \langle
E_\text{pot} \rangle
 \end{equation}
her.
Die kinetischen und potentiellen Energien lassen sich über die Annahme herleiten, dass Atome
wie harmonische Oszillatoren schwingen.
Die rücktreibene Kraft der Auslenkung eines Atoms lässt sich in guter Näherung über das Hooksche Gesetz beschreiben:
\begin{equation}
  F_R= Dx
\end{equation}
Während seine Auslenkung sich nach Newton mit
\begin{equation}
  F_T=   m \frac{d^2 x}{dt^2}
\end{equation}
beschreiben lässt.
Daraus folgt die Bewegungsgleichung:
\begin{equation}
  m \frac{d^2 x}{dt^2}+Dx=0 
\end{equation}
Welche die bekannte Lösung
\begin{equation}
  \label{eq:loes1}
  x(t)=A \cos{\frac{2\pi}{\uptau}t}
\end{equation}
mit
\begin{equation}
  \label{eq:loes2}
  \uptau^2 = 4 \pi^2 \frac{m}{D}
\end{equation}
hat, wobei $\uptau$ die Periode der Schwingung ist.
Durch Integration der rücktreibenden Kraft erhält man die potentielle Energie der Form:
\begin{equation}
  E_\text{pot}(t) = \frac{1}{2}Dx^2(t)
\end{equation}
Mit dieser ergibt sich eine mittlere kinetische Energie von:
\begin{equation}
  \langle E_\text{pot} \rangle = \frac{1}{\uptau} \int_0^\uptau \frac{1}{2}Dx(t)\symup{dt}=\frac{DA^2}{2\uptau}\int_0^\uptau \cos^2{\frac{2\pi}{\uptau}t}\symup{dt} =\frac{1}{4}DA^2
\end{equation}
Analog folgt aus den Gleichungen \eqref{eq:loes1} und \eqref{eq:loes2} die mittlere kinetische Energie:
\begin{equation}
  \langle E_\text{kin} \rangle = \frac{m}{2\uptau} A^2 \int_0^\uptau \left(\frac{2\pi}{\uptau}\right)^2 \sin^2{\frac{2\pi}{\uptau}t} \symup{dt} = \frac{1}{4}DA^2
\end{equation}
Somit folgt für die innere Energie:
\begin{equation}
  \langle u \rangle = 2 \langle E_\text{kin} \rangle
\end{equation}
Es folgt das Äquipartitionstheorem, welches besagt, dass ein Atom die mittlere kinetische Energie
\begin{equation}
  \langle E_\text{kin} \rangle = \frac{1}{2}kT
\end{equation}
pro Freiheitsgrad besitzt.
Für ein Mol eines Stoffes mit $N_l=6 \cdot 10^23/\si[per-mode=reciprocal]{\per\mol}$\cite{v201} Atomen ergibt sich eine mittlere innere Energie von
\begin{equation}
  \langle U \rangle = \frac{M}{m} \langle u \rangle = N_l k T = R T
\end{equation}
pro Freiheitsgrad.
Ein Atom im Festkörper besitzt drei Freiheitsgrade, also gilt:
\begin{equation}
  \langle U \rangle _\text{Festkörper}= 3 R T
\end{equation}
Aus den so aufgestellten Gleichungen für die Energien folgt mit Gleichung \eqref{eq:molkap} der Ausdruck:
\begin{equation}
\label{eqn:peitsche}
C_{m,V} = 3R
\end{equation}
\subsection{Abweichung vom Dulong-Petitschen Gesetz}
Der Wert $3R$ wird bei hohen Temperaturen tatsächlich erreicht.
Die meisten Elemente erreichen diesen schon bei Raumtemperatur.
Wohingegen sich dieser Wert für Substanzen mit geringem Atomgewicht erst bei
Größenordnungen um $1000\si{\celsius}$ einstellt.
Dies lässt sich nicht mit der klassischen Physik erklären,
daher muss das Problem quantenmechanisch betrachtet werden.
Die Annahme des Dulong-Petitsche Gesetzes, dass atomare Oszillatoren in Festkörpern
beliebig kleine Energie-Beträge aufnehmen und abgeben können widerspricht den Erkenntnissen der Quantenmechanik.
Diese besagt, dass Oszillatoren ihre Energie nur in festen Beträgen ändern können:
\begin{equation}
\symup{\Delta} u = n \hbar \omega
\end{equation}
Mithilfe der Boltzmann-Verteilung kommt man zu folgender Aussage über die innere Energie:
\begin{equation}
    \langle u_\text{qu} \rangle = \frac{\hbar\omega}{k_BT}\int_{\hbar\omega}^{2\hbar\omega}\exp{\left(\frac{-E}{k_BT}\right)}\symup{dE} +
                            \frac{2\hbar\omega}{k_BT}\int_{2\hbar\omega}^{3\hbar\omega}\exp{\left(\frac{-E}{k_BT}\right)}\symup{dE} + …
\end{equation}
Es ergibt sich schließlich:
\begin{equation}
    \langle U_\text{qu} \rangle = 3 N_L \langle u_\text{qu} \rangle = \frac{3 N_L \hbar
        \omega}{\exp{\left(\frac{\hbar\omega}{k_BT}\right)}-1}
\end{equation}
Aus der Taylorentwicklung der Exponentialfunktion kann entnommen werden,
dass $\langle U_\text{qu} \rangle$ für hohe Werte von $T$ gegen $3RT$ strebt.
