\section{Zielsetzung}
Zielsetzung dieses Versuchs ist es die Gültigkeit des Dulong-Petitsche-Gesetzes zu überprüfen.
\section{Theorie}
\label{sec:Theorie}
Wenn ein Körper sich um die Temperatur $\symup{\Delta} T$ erwärmt, nimmt er die Wärmemenge
\begin{equation}
\symup{\Delta} Q = m c \symup{\Delta} T
\end{equation}
auf.
Die spezifische Wärmekapazität wird durch
\begin{equation}
c_k = \frac{\symup{\Delta} Q}{m \cdot T}
\end{equation}
berechnet.
Es wird unterschieden zwischen der Wärmekapazität $C_P$ bei kontantem Druck und $C_V$ bei konstantem Volumen.
\subsection{Das Dulong-Petitsche Gesetz}
Das Dulong-Petitsche Gesetz besagt, dass die Atomwärme $C_V$ im festen Aggregatzustand
unabhängig vom chemischen Charakter des Elements konstant $3R$ beträgt.
$R$ ist hier die allgemeine Gaskonstante.
In der klassischen Physik leitet sich diese Annahme über die innere Energie
 \begin{equation}
 \langle u \rangle = \frac{1}{\tau} \int_0^\tau u(t)\symup{dt} = \langle E_\text{kin} \rangle + \langle
E_\text{pot} \rangle
 \end{equation}
her.
Die kinetische und potentielle Energien lassen sich über die Annahme herleiten, dass Atome
wie harmonische Oszillatoren schwingen.
Die rücktreibene Kraft der Auslenkung eines Atoms lässt sich in guter Näherung über das Hooksche Gesetz beschreiben:
\begin{equation}
  F_R= Dx .
\end{equation}
Whärend seine Auslenkung sich nach Newton mit
\begin{equation}
  F_T=   m \frac{d^2 x}{dt^2}
\end{equation}
beschreiben lässt.
Daraus folgt die Bewegungsgleichung:
\begin{equation}
  m \frac{d^2 x}{dt^2}+Dx=0 .
\end{equation}
Welche die bekannte Lösung
\begin{equation}
  x(t)=A \cos{2\pi t}
\end{equation}
hat.
Durch Integratio der Rücktreibenden Kraft erhält man die potentielle Energie der Form:
\begin{equation}
  \langle E_\text{pot} \rangle
\end{equation}
Aus den so aufgestellten Gleichungen für die Energien folgt der Ausdruck:
\begin{equation}
\label{eqn:peitsche}
C_V = 3R
\end{equation}
\subsection{Abweichung vom Dulong-Petitsche Gesetz}
Der Wert $3R$ wird bei hohen Temperaturen tatsächlich ereicht.
Die meisten Elemente erreichen diesen schon bei Raumtemperatur.
Substanzen mit geringem Atomgewicht erreichen diesen erst bei
Größenordnungen um $1000\si{\celsius}$.
Dies lässt sich nicht mit der klassischen Physik erklären,
daher muss das Problem quantenmechanisch betrachtet werden.
Die Annahme des Dulong-Petitsche Gesetzes, dass atomare Oszillatoren in Festkörpern
beliebig kleine Energie-Beträge aufnehmen und abgeben können widerspricht der Quantenmechanik.
Diese besagt, dass Oszillatoren ihre Energie nur in festen Beträgen ändern können:
\begin{equation}
\symup{\Delta} u = n \hbar \omega
\end{equation}
Somit lässt sich über die Energie folgende Aussage treffen:
\begin{equation}
\langle U_\text{qu} \rangle = 3 N_L \langle u_\text{qu} \rangle = \frac{3 N_L h
\omega}{e^{\frac{h\omega}{k_\text{B}T}}-1}
\end{equation}
Aus der Taylorentwicklung der Exponentialfunktion kann entnommen werden,
dass $\langle U_{qu} \rangle$ für hohe Werte von $T$ gegen $3RT$ strebt.
