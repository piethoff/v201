\section{Zielsetzung}
Zielsetzung dieses Versuchs ist es die Gültigkeit des Dulong-Petitsche-Gesetzes zu üperprüfen.
\section{Theorie}
\label{sec:Theorie}
Wenn ein Körper sich um die Temperatur $\symup{\Delta} T$ erwärmt, nimmt er die Wärmemenge
\begin{equation}
\symup{\Delta} Q = m c \symup{\Delta} T
\end{equation}
auf.
Die spezifiche Wärmekapazität wird durch
\begin{equation}
c_k = \frac{\symup{\Delta} Q}{m \cdot T}
\end{equation}
errechnet.
Es wird unterschieden zwischen der Wärmekapazität $C_P$ bei kontanten Druck und $C_V$ bei konstanten Volumen.
\subsection{Das Dulong-Petitsche Gesetz}
Das Dulong-Petitsche Gesetz besagt, dass die Atomwärme $C_V$ im festen Aggregatszustand
unabhängig vom chemischen Charakter des Elements konstant $3R$ beträgt.
$R$ ist hier die allgemeine Gaskonstante.
In der klassischen Physik lässt sich diese Annahme über die innere Energie
 \begin{equation}
 \langle u \rangle = \frac{1}{\tau} \int_0^\tau u(t)\symup{dt} = \langle E_\text{kin} \rangle + \langle
E_\text{pot} \rangle
 \end{equation}
hergeleitet.
Die kinetische und potentielle Energien lassen sich über die Annahme herleiten, dass Atome
wie harmonische Oszillatoren schwingen.
Aus den so aufgestellten Gleichungen für die Energien folgt der Ausdruck:
\begin{equation}
\label{eqn:peitsche}
C_V = 3R
\end{equation}
\subsection{Abweichung vom Dulong-Petitsche Gesetz}
Der Wert $3R$ wird bei hohen Temperaturen tatsächlich gemessen.
Die meisten Elemente erreichen diesen schon bei Raumtemperatur.
Substanzen mit geringen Atomgewicht erreichen diesen erst bei
Größenordnungen um $1000\si{\celsius}$.
Dies lässt sich nicht mit der klassischen Physik erklären,
daher muss das Problem quantenmechanisch betrachtet werden.
Die Annahme des Dulong-Petitsche Gesetzes, dass atomare Oszillatoren in Festkörpern
beliebig kleine Energien aufnehmen und abgeben können widerspricht der Quantenmechanik.
Diese besagt, dass Oszillatoren ihre Energie nur in festen Beträgen ändern können:
\begin{equation}
\symup{\Delta} u = n \hbar \omega
\end{equation}
Somit lässt sich über die Energie folgende Aussage treffen:
\begin{equation}
\langle U_\text{qu} \rangle = 3 N_L \langle u_\text{qu} \rangle = \frac{3 N_L h
\omega}{e^{\frac{h\omega}{k_\text{B}T}}-1}
\end{equation}
Aus der Taylorentwicklung der Exponentialfunktion kann entnommen werden,
dass $\langle U_{qu} \rangle$ für hohe Werte von $T$ gegen $3RT$ strebt.
\subsection{Fehlerrechnung}
Der Mittelwert errechnet sich durch:
\begin{equation}
\label{eq:mittel}
\overline{x}=\frac{1}{N}\sum \limits_{i=1}^N x_i
\end{equation}
Die Standardabweichung mit zufälligen fehlerbehafteten Werten $v_j$ mit $j=1,...,N$ lässt sich durch die Gleichung
\begin{equation}
\label{eq:standard}
s_x=\sqrt{\frac{1}{N-1}\sum \limits_{i=1}^N \left(x_i-\overline{x}\right)^2}
\end{equation}
berechnen.
Die Streuung der Mittelwerte errechnet sich wie folgt:
\begin{equation}
\label{eq:streuung}
\sigma_x=\frac{s_x}{\sqrt{\!N\,}}=\sqrt{\frac{1}{N(N-1)}\sum \limits_{i=1}^N \left(x_i-\overline{x}\right)^2}
\end{equation}
