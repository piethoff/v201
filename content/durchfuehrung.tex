\section{Durchführung}
\label{sec:Durchführung}
Um die Gültigkeit des Dulong-Petitschen Gesetzes zu bestimmen, werden die Wärmekapazitäten 
verschiedener Materialien
mithilfe eines Mischungskalorimeters bestimmt.
Hierbei wird das Material zunächst erwärmt und danach in ein Dewar-Gefäß mit Wasser eingetaucht,
bis sich ein Gleichgewicht der Temperaturen einstellt.
Es werden die Temperaturen des Wassers vor und nach dem Eintauchen und die Temperatur des Materials vor dem Eintauchen gemessen.
Unter der Annahme, dass keine Wärme nach außen verloren geht,
muss die Abkühlung des Materials eine Erwärmung des Wassers und des Gefäßes zufolge haben.
Es folgt daraus:
\begin{gather}
    c_km_k(T_k-T_m) = (c_wm_w + c_gm_g)(T_m - T_w) \\
    c_k = \frac{(c_wm_w + c_gm_g)(T_m - T_w)}{m_k(T_k - T_m)}
\end{gather}
$T_k$ ist hierbei die Temperatur des Materials vor dem Eintauchen, $T_w$ ist die des Wassers vor dem Eintauchen,
$T_m$ ist die Mischtemperatur und $c_gm_g$ ist die Wärmekapazität des Gefäßes.
Zunächst wird dafür die Wärmekapazität des Gefäßes vermessen. Dies folgt dem gleichen 
Verfahren, jedoch wird ein Material verwendet
dessen Wärmekapazität bereits bekannt ist, hier Wasser.
Es ergibt sich die Formel:
\begin{equation}
    \label{eqn:gefaess}
    c_gm_g = \frac{c_wm_y\left(T_y - T^{\prime}_m\right) - c_wm_x\left(T^{\prime}_m - T_x\right)}{T^{\prime}_m - T_x}
\end{equation}
Es ist außerdem zu beachten, dass die Theorie sich, wie in Gleichung \eqref{eqn:peitsche},
auf die Wärmekapazität bei konstantem Volumen bezieht.
Dies ist jedoch technisch sehr aufwendig umzusetzen. Stattdessen wird bei konstantem Druck 
gemessen,
wodurch quasi kein zusätzlicher Aufwand entsteht, und $C_V$ mit folgender Formel bestimmt:
\begin{equation}
    \label{eqn:dubbi}
    C_V = C_P - (3\alpha)^2\kappa\symup{V_0}T
\end{equation}
Hier ist $\alpha$ ein linearer Ausdehnungskoeffitient, $\kappa$ der Kompressionsmodul und $\symup{V_0}$ das Molvolumen.
Alle benötigten Konstanten sind im Folgenden aufgeführt:
\begin{table}
    \centering
    \caption{Materialkonstanten\cite{v201}.}
    \label{tab:const}
    \begin{tabular}{c S[table-format=2.2(0)e0] S[table-format=3.1(0)e0] S[table-format=2.1] S[table-format=3.1(0)e0] }
        \toprule
        {Material} & {$ρ/\si{\gram\per\cm\cubed}$} & {$M/\si{\gram\per\mol}$} & {$α/\SI{e-6}{\per\kelvin}$} & {$κ/\SI{e9}{\newton\per\meter\squared}$} \\
        \midrule
        {Blei} & 11.35 & 207.2 & 29.0  & 42.0  \\
        {Kupfer} & 8.96  & 63.5  & 16.8  & 136.0     \\
        {Aluminium} & 2.7   & 27.0  & 23.5  & 75.0  \\
        \bottomrule
    \end{tabular}
\end{table}
